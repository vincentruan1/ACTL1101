\documentclass[12pt,a4paper,oneside,reqno]{article}
%\usepackage{lipsum}
\usepackage{geometry}
\usepackage[utf8]{inputenc}
\usepackage [T1]{fontenc}
\usepackage{libertine}
\usepackage{libertinust1math}
\usepackage{microtype}
\usepackage{enumitem}
\usepackage{graphicx,float}
\title{ACTL1101 Group Assignment}
\author{Airfun Javam, Joel Huang, Polly Liang, Vania Valentina, Vincent Ruan.}
\date{}

\begin{document}
	\maketitle
	\section{Introduction}
	Insert introduction here.
	\subsection{Population Density}
	Population density refers to the number of people per square kilometre. That is, the lower the population density, the fewer number of people per square kilometre. Similarly, the higher the number, the more number of people per square kilometre. Whilst overcrowding may present a problem, underpopulated cities present a larger problem in the form of access to timely and adequate health care.
	
	In cities of low population density, residents may need to travel long distances to access health and medical services, thus impacting the timeliness of timely health care. This is reflected in the higher number of standardised deaths in areas of low population density of 6.7\%, compared to 5.5\% in areas of high population density. A low population density is also associated with the rural nature of an area. Talk about economics of scale here.
	
	Whilst residents of medium or high population density are able to access more timely medical services due to well established public transport and infrastructure, they however face a larger issue of long waiting times and insufficient medical staff per residence.
	
	\subsection{Internet Access}
	Rural areas are often associated with lower public and private investment and in turn more expensive access to internet
	
	In the modern society, the communication of information 
	
	Access to jobs, education,
\end{document}