\documentclass[12pt,a4paper,oneside,reqno]{article}
%\usepackage{lipsum}
\usepackage{geometry}
\usepackage[utf8]{inputenc}
\usepackage [T1]{fontenc}
\usepackage{libertine}
\usepackage{libertinust1math}
\usepackage{microtype}
\usepackage{enumitem}
\usepackage{graphicx,float}
\title{ACTL1101 Group Assignment}
\author{Airfun Javam, Joel Huang, Polly Liang,\\ Vania Valentina, Vincent Ruan.}
\date{}

\begin{document}
	\maketitle
	\section{Introduction}
	Insert introduction here.
	\subsection{Population Density}
	Population density is a measure of the number of residents residing in a given area. A low population density is often associated with its rural nature, where access to health care services may be severely impeded. 
	
	Due to geographical remoteness, residents may need to travel longer distances to access medical services such as emergency care and primary care. Access to healthcare is a particular important factor in reducing population mortality
	\begin{itemize}
		\item Improving physical, mental and social health,
		\item Detection and prevention of illnesses,
		\item Treatment of preventable diseases.
	\end{itemize}

	This is reflected in 6.7\% standardised deaths in areas of low population density, compared to 5.5\% in areas of high population density. A low population density is also associated with the rural nature of an area. Talk about economics of scale here.
	
	Whilst residents of medium or high population density are able to access more timely medical services due to well established public transport and infrastructure, they however face a larger issue of long waiting times and insufficient medical staff per residence.
	
	Governments have responded to the shortages of doctors in rural areas by introducing a Medical Rural Bonded Scholarship (MRBS) Scheme. However, the ex
	\subsection{Internet Access}
	Rural areas are often associated with lower public and private investment and in turn more expensive access to internet
	
	In the modern society, the communication of information 
	
	Access to jobs, education,
\end{document}